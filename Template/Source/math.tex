\chapter{Math}
\section{int64相乘取模\ \small(Durandal)}
	\inputminted{cpp}{Math/int64_multiply_mod.cpp}
\section{ex-Euclid\ \small(gy)}
	\inputminted{cpp}{Math/extend_gcd.cpp}
\section{中国剩余定理\ \small(Durandal)}
	返回是否可行,余数和模数结果为$ r_1 $, $ m_1 $
	\inputminted{cpp}{Math/chinese_remainder_theorem.cpp}
\section{线性同余不等式\ \small(Durandal)}
	必须满足$ 0 \leq d < m $, $ 0 \leq l \leq r < m $,返回$ \min\lbrace x \geq 0 \mid l \leq x \cdot d \bmod m \leq r \rbrace $,无解返回$ -1 $ 
	\inputminted{cpp}{Math/linear_congruence_inequality.cpp}
\section{组合数}
\section{高斯消元\ \small(ct)}
	增广矩阵大小为$ m \times (n + 1) $
	\inputminted{cpp}{Math/gauss_elimination.cpp}
\section{Miller\ Rabin\ \&\ Pollard\ Rho\ \small(gy)}
	In Java, use BigInteger.isProbablePrime(int certainty) to replace miller\_rabin(BigInteger number)\\
	\begin{tabular}{l r}
		\hline
		Test Set & First Wrong Answer\\\hline
		$ 2 $ & $ 2047 $\\\hline
		$ 2, 3 $ & $ 1,373,653 $\\\hline
		$ 31, 73 $ & $ 9,080,191 $\\\hline
		$ 2, 3, 5 $ & $ 25,326,001 $\\\hline
		$ 2, 3, 5, 7 $ & (INT32\_MAX)$ 3,215,031,751 $\\\hline
		$ 2, 7, 61 $ & $ 4,759,123,141 $\\\hline
		$ 2, 13, 23, 1662803 $ & $ 1,122,004,669,633 $\\\hline
		$ 2, 3, 5, 7, 11 $ & $ 2,152,302,898,747 $\\\hline
		$ 2, 3, 5, 7, 11, 13 $ & $ 3,474,749,660,383 $\\\hline
		$ 2, 3, 5, 7, 11, 13, 17 $ & $ 341,550,071,728,321 $\\\hline
		$ 2, 3, 5, 7, 11, 13, 17, 19, 23 $ & $ 3,825,123,056,546,413,051 $\\\hline
		$ 2, 3, 5, 7, 11, 13, 17, 19, 23, 29, 31, 37 $ & (INT64\_MAX)$ 318,665,857,834,031,151,167,461 $\\\hline
		$ 2, 3, 5, 7, 11, 13, 17, 19, 23, 29, 31, 37, 41 $ & $ 3,317,044,064,679,887,385,961,981 $\\\hline
	\end{tabular}
	\inputminted{cpp}{Math/miller_rabin_and_pollard_rho.cpp}
\section{$ O(m ^ 2 \log n) $线性递推\ \small(lhy)}
	\inputminted{cpp}{Math/linear_rec.cpp}
\section{线性基\ \small(ct)}
	\inputminted{cpp}{Math/linear_base.cpp}
\section{FFT\ NTT\ FWT\ \small(lhy,ct,gy)}
	\subsection*{FFT\ \small(ct)}
		$ 0 $-based
		\inputminted{cpp}{Math/fft.cpp}
	\subsection*{NTT\ \small(gy)}
		$ 0 $-based
		\inputminted{cpp}{Math/ntt.cpp}
	\subsection*{FWT\ \small(lhy)}
		$ 0 $-based
		\inputminted{cpp}{Math/fwt.cpp}
\section{Lagrange插值\ \small(ct)}
	求解$ \sum\limits_{i = 1}^{n} i^k \bmod (10^9 + 7)$
	\inputminted{cpp}{Math/lagrange_polynomial.cpp}
\section{杜教筛\ \small(ct)}
	\inputminted{cpp}{Math/du_jiao_sieve.cpp}
\section{BSGS\ \small(ct,Durandal)}
	\subsection{BSGS(ct)}
		$ p $是素数,返回$ \min\lbrace x \geq 0 \mid y^x \equiv z \pmod p \rbrace $
		\inputminted{cpp}{Math/bsgs.cpp}
	\subsection{ex-BSGS(Durandal)}
		必须满足$ 0 \leq a < p $, $ 0 \leq b < p $,返回$ \min\lbrace x \geq 0 \mid a^x \equiv b \pmod p\rbrace $
		\inputminted{cpp}{Math/ex_bsgs.cpp}
\section{直线下整点个数\ \small(gy)}
	必须满足$ a \geq 0 $, $ b \geq 0 $, $ m > 0 $,返回$ \sum\limits_{i=0}^{n-1} \frac{a + bi}{m} $
	\inputminted{cpp}{Math/points_below_line.cpp}
\section{单纯形}
\section{辛普森积分}
\section{数学知识\ \small(gy)}
\newcommand{\eularian}{\genfrac\langle\rangle{0pt}{0}}
\newcommand{\Eularian}[2]{\left\langle\!\!\!\eularian{#1}{#2}\!\!\!\right\rangle}
	\subsection*{求和公式}
		\begin{itemize}
			\item $ \sum\limits_{k=1}^{n} (2k - 1)^2 = \frac{1}{3} n(4n^2 - 1) $
			\item $ \sum\limits_{k=1}^{n} k^3 = \frac{1}{4} n^2(n + 1)^2 $
			\item $ \sum\limits_{k=1}^{n} (2k - 1)^3 = n^2(2n^2 - 1) $
			\item $ \sum\limits_{k=1}^{n} k^4 = \frac{1}{30} n(n + 1) (2n + 1) (3n^2 + 3m - 1) $
			\item $ \sum\limits_{k=1}^{n} k^5 = \frac{1}{12} n^2(n + 1)^2(2n^2 + 2n - 1) $
			\item $ \sum\limits_{k=1}^{n} k(k + 1) = \frac{1}{3} n(n + 1)(n + 2) $
			\item $ \sum\limits_{k=1}^{n} k(k + 1)(k + 2) = \frac{1}{4} n(n + 1)(n + 2)(n + 3) $
			\item $ \sum\limits_{k=1}^{n} k(k + 1)(k + 2)(k + 3) = \frac{1}{5} n(n + 1)(n + 2)(n + 3)(n + 4) $
		\end{itemize}
	\subsection*{错排公式}
		$ D_n $表示$ n $个元素错位排列的方案数
		\\$ D_1 = 0, D_2 = 1 $
		\\$ D_n = (n - 1)(D_{n - 2} + D_{n - 1}), n \geq 3 $
		\\$ D_n = n! \cdot (1 - \frac{1}{1!} + \frac{1}{2!} - \dots + (-1)^n\frac{1}{n!}) $
	\subsection*{Fibonacci sequence}
		$ F_0 = 0, F_1 = 1 $
		\\$ F_n = F_{n - 1} + F_{n - 2} $
		\\$ F_{n + 1} \cdot F_{n - 1} - F_{n}^2 = (-1)^n $
		\\$ F_{-n} = (-1)^n F_n $
		\\$ F_{n + k} = F_k \cdot F_{n + 1} + F_{k - 1} \cdot F_n $
		\\$ \gcd(F_m, F_n) = F_{\gcd(m, n)} $
		\\$ F_m \mid F_n^2 \Leftrightarrow nF_n \mid m $
		\\$ F_n = \frac{\varphi^n - \varPsi^n}{\sqrt{5}}, \varphi = \frac{1 + \sqrt{5}}{2}, \varPsi = \frac{1 - \sqrt{5}}{2} $
		\\$ F_n = \lfloor \frac{\varphi^n}{\sqrt{5}} + \frac{1}{2} \rfloor, n \geq 0 $
		\\$ n(F) = \lfloor \log_\varphi(F \cdot \sqrt{5} + \frac{1}{2}) \rfloor $
	\subsection*{Stirling number (1st kind)}
		用$ {n \brack k} $表示Stirling number (1st kind),为将$ n $个元素分成$ k $个环的方案数
		\\$ {n + 1 \brack k} = n {n \brack k} + {n \brack k - 1}, k > 0 $
		\\$ {0 \brack 0} = 1, {n \brack 0} = {0 \brack n} = 0, n > 0 $
		\\$ {n \brack k} $为将$ n $个元素分成$ k $个环的方案数
		\\$ {x \brack x - n} = \sum\limits_{k = 0}^{n} \Eularian{n}{k} \binom{x + k}{2 n} $
	\subsection*{Stirling number (2nd kind)}
		用$ {n \brace k} $表示Stirling number (2nd kind),为将$ n $个元素划分成$ k $个非空集合的方案数
		\\$ {n + 1 \brace k} = k {n \brace k} + {n \brace k - 1}, k > 0 $
		\\$ {0 \brace 0} = 1, {n \brace 0} = {0 \brace n} = 0, n > 0 $
		\\$ {n \brace k} = \frac{1}{k!} \sum\limits_{j = 0}^{k} (-1) ^ {k - j} \binom{k}{j} j^n $
		\\$ {n \brace k} $
		\\$ {x \brace x - n} = \sum\limits_{k = 0}^{n} \Eularian{n}{k} \binom{x + n - k - 1}{2 n} $
	\subsection*{Catalan number}
		$ c_n $表示长度为$ 2n $的合法括号序的数量
		\\$ c_1 = 1 $, $ c_{n+1} = \sum\limits_{i=1}^{n} c_i \times c_{n + 1 - i} $
		\\$ c_n = \frac{\binom{2n}{n}}{n + 1} $
	\subsection*{Bell number}
		$ B_n $表示基数为$ n $的集合的划分方案数
		\\$ B_i = \begin{cases}
			1 & i = 0\\
			\sum\limits_{k = 0}^{n} \binom{n}{k} B_k & i > 0
		\end{cases} $
		\\$ B_n = \sum\limits_{k = 0}^{n} {n \brace k} $
		\\$ B_{p^m + n} \equiv m B_n + B_{n + 1} \pmod p $
	\subsection*{五边形数定理}
		$ p(n) $表示将$ n $划分为若干个正整数之和的方案数
		\\$ p(n) = \sum\limits_{k \in \mathbb{N}^\ast} (-1)^{k - 1} p(n - \frac{k(3k - 1)}{2}) $
	\subsection*{Bernoulli number}
		$ \sum\limits_{j = 0}^{m} \binom{m + 1}{j} B_j = 0, m > 0 $
		\\$ B_i = \begin{cases}
			1 & i = 0\\
			-\frac{\sum\limits_{j = 0}^{i - 1} \binom{i + 1}{j} B_j}{i + 1} & i > 0
		\end{cases} $
		\\$ \sum\limits_{k = 1}^{n} k ^ m = \frac{1}{m + 1} \sum\limits_{k = 0}^{m} \binom{m + 1}{k} B_k n ^ {m + 1 - k} $
	\subsection*{Stirling permutation}
		$ 1, 1, 2, 2 \dots , n, n $的排列中,对于每个$ i $,都有两个$ i $之间的数大于$ i $
		\\排列方案数为$ (2n - 1)!! $
	\subsection*{Eulerian number}
		$ \eularian{n}{k} $表示$ 1 $到$ n $的排列中,恰有$ k $个数比前一个大的方案数
		\\$ \eularian{n}{0} = \eularian{n}{n - 1} = 1 $
		\\$ \eularian{0}{m} = [m = 0] $
		\\$ \eularian{n}{m} = \eularian{n}{n - 1 - m} $
		\\$ \eularian{n}{m} = (m + 1) \eularian{n - 1}{m} + (n - m) \eularian{n - 1}{m - 1} $
		\\$ \eularian{n}{m} = \sum\limits_{k = 0}^{m} (-1)^k \binom{n + 1}{k} (m + 1 - k)^n $
	\subsection*{Eulerian number (2nd kind)}
		$ \Eularian{n}{k} $表示Stirling permutation中,恰有$ k $个数比前一个大的方案数
		\\$ \Eularian{n}{m} = (2n - m - 1) \Eularian{n - 1}{m - 1} + (m + 1) \Eularian{n - 1}{m} $
		\\$ \Eularian{n}{0} = 1 $
		\\$ \Eularian{0}{m} = [m = 0] $
	\subsection*{Möbius function}
		$ \mu(n) = \begin{cases}
			1 & n \text{ is a square-free positive integer with an even number of prime factors}\\
			-1 & n \text{ is a square-free positive integer with an odd number of prime factors}\\
			0 & n \text{ has a squared prime factor}
		\end{cases} $
		\\$ \sum\limits_{d \mid n} \mu(d) = \begin{cases}
			1 & n = 1\\
			0 & n > 1
		\end{cases} $
		\\$ g(n) = \sum\limits_{d \mid n} f(d) \Leftrightarrow f(n) = \sum\limits_{d \mid n} \mu(d) g(\frac{n}{d}) $
	\subsection*{Lagrange polynomial}
		给定次数为$ n $的多项式函数$ L(x) $上的$ n + 1 $个点$ (x_0, y_0), (x_1, y_1), \dots, (x_n, y_n) $
		\\则$ L(x) = \sum\limits_{j = 0}^{n} y_j \prod\limits_{0 \leq m \leq n, m \ne j} \frac{x - x_m}{x_j - x_m} $