\chapter{Graph Theory}
\section{2-SAT\ \small(ct)}
	\inputminted{cpp}{GraphTheory/2_sat.cpp}
\section{割点与桥\ \small(ct)}
	\subsection*{割点}
		\inputminted{cpp}{GraphTheory/cut_point.cpp}
	\subsection*{桥}
		\inputminted{cpp}{GraphTheory/bridge.cpp}
\section{Steiner\ tree\ \small(lhy)}
	\inputminted{cpp}{GraphTheory/steiner_tree.cpp}
\section{K短路\ \small(lhy)}
	\inputminted{cpp}{GraphTheory/kth_minimum_path.cpp}
\section{最大团}
\section{一般图最大匹配}
\section{KM算法\ \small(Nightfall)}
	$ O(n^3) $,$ 1 $-based,最大权匹配
	\\不存在的边权值开到$ -n \times (\left| MAXV \right|) $,$ \infty $为$ 3 n \times (\left| MAXV \right|) $
	\\匹配为$ (lk_i, i) $
	\inputminted{cpp}{GraphTheory/km.cpp}
\section{支配树\ \small(Nightfall,ct)}
	\subsection*{DAG\ \small(ct)}
		\inputminted{cpp}{GraphTheory/dominator_tree_dag.cpp}
	\subsection*{一般图\ \small(Nightfall)}
		\inputminted{cpp}{GraphTheory/dominator_tree.cpp}
\section{虚树\ \small(ct)}
	\inputminted{cpp}{GraphTheory/virtual_tree.cpp}
\section{点分治\ \small(ct)}
	\inputminted{cpp}{GraphTheory/divide_conquer_on_tree.cpp}
\section{树上倍增\ \small(ct)}
	\inputminted{cpp}{GraphTheory/multiplier_on_tree.cpp}
\section{树分块}
\section{Prufer编码}
\section{Link-Cut\ Tree\ \small(ct)}
	\inputminted{cpp}{GraphTheory/link_cut_tree.cpp}
\section{圆方树\ \small(ct)}
	\inputminted{cpp}{GraphTheory/circle_square_tree.cpp}
\section{最小割}
\section{最大流\ \small(ct)}
	\inputminted{cpp}{GraphTheory/dinic.cpp}
\section{费用流\ \small(ct)}
	\subsection*{Dinic(ct)}
		\inputminted{cpp}{GraphTheory/min_cost_max_flow.cpp}
	\subsection*{zkw(lhy)}
		\inputminted{cpp}{GraphTheory/zkw_min_cost_flow.cpp}
\section{有上下界的网络流\ \small(Durandal)}
	$ B(u, v) $表示边$ (u, v) $流量的下界,$ C(u, v) $表示边$ (u, v) $流量的上界,设$ F(u, v) $表示边$ (u, v) $的实际流量\\
	设$ G(u, v) = F(u, v) - B(u, v) $,则$ 0 \leq G(u, v) \leq C(u, v) - B(u, v) $
	\begin{itemize}
		\item 无源汇的上下界可行流
			\\建立超级源点$ S^\ast $和超级汇点$ T^\ast $,对于原图每一条边$ (u, v) $在新网络中连如下三条边:$ S^\ast \to v $,容量为$ B(u, v) $;$ u \to T^\ast $,容量为$ B(u, v) $;$ u \to v $,容量为$ C(u, v) - B(u, v) $。最后求新网络的最大流,判断从超级源点$ S^\ast $出发的边是否都满流即可,边$ (u, v) $的最终解中的实际流量为$ G(u, v) + B(u, v) $。
		\item 有源汇的上下界可行流
			\\从汇点$ T $到源点$ S $连一条上界为$ \infty $,下界为$ 0 $的边。按照无源汇的上下界可行流一样做即可,流量即为$ T \to S $边上的流量。
		\item 有源汇的上下界最大流
			\begin{itemize}
				\item 在有源汇的上下界可行流中,从汇点$ T $到源点$ S $的边改为连一条上界为$ \infty $,下界为$ x $的边。$ x $满足二分性质,找到最大的$ x $使得新网络存在有源汇的上下界可行流即为原图的最大流。
				\item 从汇点$ T $到源点$ S $连一条上界为$ \infty $,下界为$ 0 $的边,变成无源汇的网络。按照无源汇的上下界可行流的方法,建立超级源点$ S^\ast $与超级汇点$ T^\ast $,求一遍$ S^\ast \to T^\ast $的最大流,再将从汇点$ T $到源点$ S $的这条边拆掉,求一次$ S \to T $的最大流即可。
			\end{itemize}
		\item 有源汇的上下界最小流
			\begin{itemize}
				\item 在有源汇的上下界可行流中,从汇点$ T $到源点$ S $的边改为连一条上界为$ x $,下界为$ 0 $的边。$ x $满足二分性质,找到最小的$ x $使得新网络存在有源汇的上下界可行流即为原图的最大流。
				\item 按照无源汇的上下界可行流的方法,建立超级源点$ S^\ast $与超级汇点$ T^\ast $,求一遍$ S^\ast \to T^\ast $的最大流,但是注意不加上汇点$ T $到源点$ S $的这条边,即不使之改为无源汇的网络去求解。求完后,再加上那条汇点$ T $到源点$ S $的边,上界为$ \infty $的边。因为这条边的下界为$ 0 $,所以$ S^\ast $, $ T^\ast $无影响,再求一次$ S^\ast \to T^\ast $的最大流。若超级源点$ S^\ast $出发的边全部满流,则$ T \to S $边上的流量即为原图的最小流,否则无解。
			\end{itemize}
	\end{itemize}
\section{差分约束}
\section{图论知识\ \small(gy,lhy)}
	\subsection*{弦图}
		弦图:任意点数$ \geq 4 $的环皆有弦的无向图
		\\单纯点:与其相邻的点的诱导子图为完全图的点
		\\完美消除序列:每次选择一个单纯点删去的序列
		\\弦图必有完美消除序列
		\\$ O(m + n) $求弦图的完美消除序列:每次选择未选择的标号最大的点,并将与其相连的点标号$ + 1 $,得到完美消除序列的反序
		\\最大团数$ = $最小染色数:按完美消除序列从后往前贪心地染色
		\\最小团覆盖$ = $最大点独立集:按完美消除序列从前往后贪心地选点加入点独立集
	\subsection*{计数问题}
		\begin{itemize}
			\item \textbf{有根树计数}
			\\$ a_1 = 1 $
			\\$ a_{n + 1} = \frac{\sum\limits_{j = 1}^{n} j \cdot a_j \cdot S_{n, j}}{n} $
			\\$ S_{n, j} = \sum\limits_{i = 1}^{n / j} a_{n + 1 - ij} = S_{n - j, j} + a_{n + 1 - j} $
			\item \textbf{无根树计数}
			\\$ \begin{cases}
				a_n - \sum\limits_{i = 1}^{n / 2} a_i a_{n - i} & n \text{ is odd}\\
				a_n - \sum\limits_{i = 1}^{n / 2} a_i a_{n - i} + \frac{1}{2} a_{\frac{n}{2}} (a_{\frac{n}{2}} + 1) & n \text{ is even}
			\end{cases} $
			\item \textbf{完全图生成树计数}
			\\$ n^{n - 2} $
			\item \textbf{矩阵-树定理}
			\\设$ \mathbf{A}\lbrack G \rbrack $为图$ G $的邻接矩阵、$ \mathbf{D}\lbrack G \rbrack $为图$ G $的度数矩阵,则图$ G $的不同生成树的个数为 $ \mathbf{C}\lbrack G \rbrack = \mathbf{D}\lbrack G \rbrack - \mathbf{A}\lbrack G \rbrack $的任意一个$ n - 1 $阶主子式的行列式值。
			\item \textbf{偶数点完全图完备匹配计数}
			\\$ (n - 1)!! $
			\item \textbf{无根二叉树计数}
			\\$ (2n - 5)!! $
			\item \textbf{有根二叉树计数}
			\\$ (2n - 3)!! $
		\end{itemize}
	\subsection*{最大权闭合子图}
		给定一个带点权的有向图,求其最大权闭合子图。
		\\从源点$ S $向每一条正权点连一条容量为权值的边,每个负权点向汇点$ T $连一条容量为权值绝对值的边,有向图原来的边容量为$ \infty $。求它的最小割,与源点$ S $连通的点构成最大权闭合子图,权值为\textit{正权值和}$ - $\textit{最小割}。
	\subsection*{最大密度子图}
		给定一个无向图,求其一个子图,使得子图的边数$ \left| E \right| $和点数$ \left| V \right| $满足$ \frac{\left| E \right|}{\left| V \right|} $最大。
		\\二分答案$ k $,使得$ \left| E \right| - k \left| V \right| \geq 0 $有解,将原图边和点都看作点,边$ (u, v) $分别向$ u $和$ v $连边求最大权闭合子图。
		