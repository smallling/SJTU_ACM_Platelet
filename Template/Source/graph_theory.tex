\chapter{Graph Theory}
\section{2-SAT}
\section{双连通分量}
\subsection{点双连通分量}
\subsection{边双连通分量}
\section{K短路\ \small(lhy)}
\inputminted{cpp}{GraphTheory/kth_minimum_path.cpp}
\section{最大团}
\section{一般图最大匹配}
\section{树}
\subsection{虚树}
\subsection{矩阵树定理}
\subsection{点分治}
\subsection{Prufer编码}
\subsection{Link-Cut\ Tree\ \small(ct)}
\inputminted{cpp}{GraphTheory/link_cut_tree.cpp}
\subsection{树上倍增}
\subsection{数链剖分}
\section{仙人掌}
\section{带花树}
\section{KM算法}
\section{支配树}
\subsection{DAG}
\subsection{一般图}
\section{弦图}
\section{网络流}
\section{最小割}
\section{最大流}
\section{费用流}
\section{有上下界的网络流\ \small(Durandal)}
$ B(u, v) $表示边$ (u, v) $流量的下界,$ C(u, v) $表示边$ (u, v) $流量的上界,设$ F(u, v) $表示边$ (u, v) $的实际流量\\
设$ G(u, v) = F(u, v) - B(u, v) $,则$ 0 \leq G(u, v) \leq C(u, v) - B(u, v) $
\begin{itemize}
	\item 无源汇的上下界可行流\\
	建立超级源点$ S^\ast $和超级汇点$ T^\ast $,对于原图每一条边$ (u, v) $在新网络中连如下三条边:$ S^\ast \to v $,容量为$ B(u, v) $;$ u \to T^\ast $,容量为$ B(u, v) $;$ u \to v $,容量为$ C(u, v) - B(u, v) $。最后求新网络的最大流,判断从超级源点$ S^\ast $出发的边是否都满流即可,边$ (u, v) $的最终解中的实际流量为$ G(u, v) + B(u, v) $。
	\item 有源汇的上下界可行流\\
	从汇点$ T $到源点$ S $连一条上界为$ \infty $,下界为$ 0 $的边。按照无源汇的上下界可行流一样做即可,流量即为$ T \to S $边上的流量。
	\item 有源汇的上下界最大流
	\begin{itemize}
		\item 在有源汇的上下界可行流中,从汇点$ T $到源点$ S $的边改为连一条上界为$ \infty $,下界为$ x $的边。$ x $满足二分性质,找到最大的$ x $使得新网络存在有源汇的上下界可行流即为原图的最大流。
		\item 从汇点$ T $到源点$ S $连一条上界为$ \infty $,下界为$ 0 $的边,变成无源汇的网络。按照无源汇的上下界可行流的方法,建立超级源点$ S^\ast $与超级汇点$ T^\ast $,求一遍$ S^\ast \to T^\ast $的最大流,再将从汇点$ T $到源点$ S $的这条边拆掉,求一次$ S \to T $的最大流即可。
	\end{itemize}
	\item 有源汇的上下界最小流
	\begin{itemize}
		\item 在有源汇的上下界可行流中,从汇点$ T $到源点$ S $的边改为连一条上界为$ x $,下界为$ 0 $的边。$ x $满足二分性质,找到最小的$ x $使得新网络存在有源汇的上下界可行流即为原图的最大流。
		\item 按照无源汇的上下界可行流的方法,建立超级源点$ S^\ast $与超级汇点$ T^\ast $,求一遍$ S^\ast \to T^\ast $的最大流,但是注意不加上汇点$ T $到源点$ S $的这条边,即不使之改为无源汇的网络去求解。求完后,再加上那条汇点$ T $到源点$ S $的边,上界为$ \infty $的边。因为这条边的下界为$ 0 $,所以$ S^\ast $, $ T^\ast $无影响,再求一次$ S^\ast \to T^\ast $的最大流。若超级源点$ S^\ast $出发的边全部满流,则$ T \to S $边上的流量即为原图的最小流,否则无解。
	\end{itemize}
\end{itemize}
\subsection{zkw费用流}
\section{差分约束}