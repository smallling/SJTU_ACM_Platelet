\begin{itemize}
    \item \textbf{求和公式}
        \begin{itemize}
            \item $ \sum\limits_{k=1}^{n} (2k - 1)^2 = \frac{1}{3} n(4n^2 - 1) $
            \item $ \sum\limits_{k=1}^{n} k^3 = \frac{1}{4} n^2(n + 1)^2 $
            \item $ \sum\limits_{k=1}^{n} (2k - 1)^3 = n^2(2n^2 - 1) $
            \item $ \sum\limits_{k=1}^{n} k^4 = \frac{1}{30} n(n + 1) (2n + 1) (3n^2 + 3m - 1) $
            \item $ \sum\limits_{k=1}^{n} k^5 = \frac{1}{12} n^2(n + 1)^2(2n^2 + 2n - 1) $
            \item $ \sum\limits_{k=1}^{n} k(k + 1) = \frac{1}{3} n(n + 1)(n + 2) $
            \item $ \sum\limits_{k=1}^{n} k(k + 1)(k + 2) = \frac{1}{4} n(n + 1)(n + 2)(n + 3) $
            \item $ \sum\limits_{k=1}^{n} k(k + 1)(k + 2)(k + 3) = \frac{1}{5} n(n + 1)(n + 2)(n + 3)(n + 4) $
        \end{itemize}
    \item \textbf{错排公式}
        \\$ D_n $表示$ n $个元素错位排列的方案数
        \\$ D_1 = 0, D_2 = 1 $
        \\$ D_n = (n - 1)(D_{n - 2} + D_{n - 1}), n \geq 3 $
        \\$ D_n = n! \cdot (1 - \frac{1}{1!} + \frac{1}{2!} - \dots + (-1)^n\frac{1}{n!}) $
    \item \textbf{Fibonacci sequence}
        \\$ F_0 = 0, F_1 = 1 $
        \\$ F_n = F_{n - 1} + F_{n - 2} $
        \\$ F_{n + 1} \cdot F_{n - 1} - F_{n}^2 = (-1)^n $
        \\$ F_{-n} = (-1)^n F_n $
        \\$ F_{n + k} = F_k \cdot F_{n + 1} + F_{k - 1} \cdot F_n $
        \\$ \gcd(F_m, F_n) = F_{\gcd(m, n)} $
        \\$ F_m \mid F_n^2 \Leftrightarrow nF_n \mid m $
        \\$ F_n = \frac{\varphi^n - \varPsi^n}{\sqrt{5}}, \varphi = \frac{1 + \sqrt{5}}{2}, \varPsi = \frac{1 - \sqrt{5}}{2} $
        \\$ F_n = \lfloor \frac{\varphi^n}{\sqrt{5}} + \frac{1}{2} \rfloor, n \geq 0 $
        \\$ n(F) = \lfloor \log_\varphi(F \cdot \sqrt{5} + \frac{1}{2}) \rfloor $
    \item \textbf{第一类Stirling number}
        \\用$ s(n, k) = (-1) ^ {n - k} {n \brack k} $表示第一类Stirling number
        \\$ {n + 1 \brack k} = n {n \brack k} + {n \brack k - 1}, k > 0 $
        \\$ {0 \brack 0} = 1, {n \brack 0} = {0 \brack n} = 0, n > 0 $
        \\$ {n \brack k} $为将$ n $个元素分成$ k $个环的方案数
    \item \textbf{第二类Stirling number}
        \\用$ S(n, k) = {n \brace k} $表示第二类Stirling number
        \\$ {n + 1 \brace k} = k {n \brace k} + {n \brace k - 1}, k > 0 $
        \\$ {0 \brace 0} = 1, {n \brace 0} = {0 \brace n} = 0, n > 0 $
        \\$ {n \brace k} = \frac{1}{k!} \sum\limits_{j = 0}^{k} (-1) ^ {k - j} \binom{k}{j} j ^ n $
        \\$ {n \brace k} $为将$ n $个元素划分成$ k $个非空集合的方案数
    \item \textbf{Catalan number}
        \\$ c_n $表示长度为$ 2n $的合法括号序的数量
        \\$ c_1 = 1 $, $ c_{n+1} = \sum\limits_{i=1}^{n} c_i \times c_{n + 1 - i} $
        \\$ c_n = \frac{\binom{2n}{n}}{n + 1} $
    \item \textbf{Bell number}
        \\$ B_n $表示基数为$ n $的集合的划分方案数
        \\$ B_i = \begin{cases}
            1 & i = 0\\
            \sum\limits_{k = 0}^{n} \binom{n}{k} B_k & i > 0
        \end{cases} $
        \\$ B_n = \sum\limits_{k = 0}^{n} {n \brace k} $
    \item \textbf{五边形数定理}
        \\$ p(n) $表示将$ n $划分为若干个正整数之和的方案数
        \\$ p(n) = \sum\limits_{k \in \mathbb{N}^\ast} (-1)^{k - 1} p(n - \frac{k(3k - 1)}{2}) $
    \item \textbf{Bernoulli number}
        \\$ \sum\limits_{j = 0}^{m} \binom{m + 1}{j} B_j = 0, m > 0 $
        \\$ B_i = \begin{cases}
            1 & i = 0\\
            -\frac{\sum\limits_{j = 0}^{i - 1} \binom{i + 1}{j}}{i + 1} & i > 0
        \end{cases} $
        \\$ \sum\limits_{k = 1}^{n} k ^ m = \frac{1}{m + 1} \sum\limits_{k = 0}^{m} \binom{m + 1}{k} B_k n ^ {m + 1 - k} $
    \item \textbf{Möbius function}
        \\$ \mu(n) = \begin{cases}
            1 & n \text{ is a square-free positive integer with an even number of prime factors}\\
            -1 & n \text{ is a square-free positive integer with an odd number of prime factors}\\
            0 & n \text{ has a squared prime factor}
        \end{cases} $
        \\$ \sum\limits_{d \mid n} \mu(d) = \begin{cases}
            1 & n = 1\\
            0 & n > 1
        \end{cases} $
        \\$ g(n) = \sum\limits_{d \mid n} f(d) \Leftrightarrow f(n) = \sum\limits_{d \mid n} \mu(d) g(\frac{n}{d}) $
    \item \textbf{Lagrange polynomial}
        \\给定次数为$ n $的多项式函数$ L(x) $上的$ n + 1 $个点$ (x_0, y_0), (x_1, y_1), \dots, (x_n, y_n) $
        \\则$ L(x) = \sum\limits_{j = 0}^{n} y_j \prod\limits_{0 \leq m \leq n, m \ne j} \frac{x - x_m}{x_j - x_m} $
    \item \textbf{树的计数}
        \begin{itemize}
            \item \textbf{有根树计数}
                \\$ a_1 = 1 $
                \\$ a_{n + 1} = \frac{\sum\limits_{j = 1}^{n} j \cdot a_j \cdot S_{n, j}}{n} $
                \\$ S_{n, j} = \sum\limits_{i = 1}^{n / j} a_{n + 1 - ij} = S_{n - j, j} + a_{n + 1 - j} $
            \item \textbf{无根树计数}
                \\$ \begin{cases}
                    a_n - \sum\limits_{i = 1}^{n / 2} a_i a_{n - i} & n \text{ is odd}\\
                    a_n - \sum\limits_{i = 1}^{n / 2} a_i a_{n - i} + \frac{1}{2} a_{\frac{n}{2}} (a_{\frac{n}{2}} + 1) & n \text{ is even}
                \end{cases} $
            \item \textbf{完全图生成树计数}
                \\$ n^{n - 2} $
            \item \textbf{矩阵-树定理}
                \\设$ \mathbf{A}\lbrack G \rbrack $为图$ G $的邻接矩阵、$ \mathbf{D}\lbrack G \rbrack $为图$ G $的度数矩阵,则图$ G $的不同生成树的个数为 $ \mathbf{C}\lbrack G \rbrack = \mathbf{D}\lbrack G \rbrack - \mathbf{A}\lbrack G \rbrack $的任意一个$ n - 1 $阶主子式的行列式值。
        \end{itemize}
    \item \textbf{Euler characteristic}
        \\ 平面图的顶点个数$ V $,边数$ E $,平面被划分的区域数$ F $,组成图形的连通部分的数目$ C $满足:
        \\$ V - E + F = C + 1 $
    \item \textbf{Pick theorem}
        \\顶点为整点的简单多边形,其面积$ A $,内部格点数$ i $,边上格点数$ b $满足:
        \\$ A = i + \frac{b}{2} - 1 $
    \item \textbf{平面几何公式}
        \begin{itemize}
            \item \textbf{三角形}
                \\半周长$ p = \frac{a + b + c}{2} $
                \\面积$ S = \frac{1}{2} a H_a = \frac{1}{2} a b \cdot \sin C = \sqrt{p(p - a)(p - b)(p - c)} = p r = \frac{a b c}{4R} $
                \\中线长$ M_a = \frac{1}{2} \sqrt{2(b^2 + c^2) - a^2} = \frac{1}{2} \sqrt{b^2 + c^2 + 2 b c \cdot \cos A} $
                \\角平分线长$ T_a = \frac{\sqrt{bc((b + c)^2 - a^2)}}{b + c} = \frac{2 b c}{b + c} \cos \frac{A}{2} $
                \\高$ H_a = b \sin C = \sqrt{b^2 - (\frac{a^2 + b^2 - c^2}{2 a})^2} $
                \\内切圆半径$ r = \frac{S}{p} = 4 R \sin \frac{A}{2} \sin \frac{B}{2} \sin \frac{C}{2} = \sqrt{\frac{(p - a)(p - b)(p - c)}{p}} = p \tan \frac{A}{2} \tan \frac{B}{2} \tan \frac{C}{2} $
                \\外接圆半径$ R = \frac{a b c}{4 S} = \frac{a}{2 \sin A} $
                \\旁切圆半径$ r_A = \frac{2 S}{- a + b + c} $
                \\重心$ \left(\frac{x_1 + x_2 + x_3}{3}, \frac{y_1 + y_2 + y_3}{3}\right) $
                \\外心$ \left(\frac{\left|\begin{array}{cccc}
                        x_1^2 + y_1^2 & y_1 & 1\\
                        x_2^2 + y_2^2 & y_2 & 1\\
                        x_3^2 + y_3^2 & y_3 & 1
                    \end{array}\right|}{2 \left|\begin{array}{cccc}
                        x_1 & y_1 & 1\\
                        x_2 & y_2 & 1\\
                        x_3 & y_3 & 1
                    \end{array}\right|}, \frac{\left|\begin{array}{cccc}
                        x_1 & x_1^2 + y_1^2 & 1\\
                        x_2 & x_2^2 + y_2^2 & 1\\
                        x_3 & x_3^2 + y_3^2 & 1
                    \end{array}\right|}{2\left|\begin{array}{cccc}
                        x_1 & y_1 & 1\\
                        x_2 & y_2 & 1\\
                        x_3 & y_3 & 1
                    \end{array}\right|}\right) $
                \\内心$ \left(\frac{a x_1 + b x_2 + c x_3}{a + b + c}, \frac{a y_1 + b y_2 + c y_3}{a + b + c}\right) $
                \\垂心$ \left(\frac{\left|\begin{array}{cccc}
                    x_2 x_3 + y_2 y_3 & 1 & y_1\\
                    x_3 x_1 + y_3 y_1 & 1 & y_2\\
                    x_1 x_2 + y_1 y_2 & 1 & y_3
                    \end{array}\right|}{2 \left|\begin{array}{cccc}
                    x_1 & y_1 & 1\\
                    x_2 & y_2 & 1\\
                    x_3 & y_3 & 1
                    \end{array}\right|}, \frac{\left|\begin{array}{cccc}
                    x_2 x_3 + y_2 y_3 & x_1 & 1\\
                    x_3 x_1 + y_3 y_1 & x_2 & 1\\
                    x_1 x_2 + y_1 y_2 & x_3 & 1
                    \end{array}\right|}{2\left|\begin{array}{cccc}
                    x_1 & y_1 & 1\\
                    x_2 & y_2 & 1\\
                    x_3 & y_3 & 1
                    \end{array}\right|}\right) $
                \\旁心$ \left(\frac{- a x_1 + b x_2 + c x_3}{- a + b + c}, \frac{- a y_1 + b y_2 + c y_3}{- a + b + c}\right) $
        \end{itemize}
\end{itemize}
